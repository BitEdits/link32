\documentclass{article}
\usepackage{url}
\usepackage{listings}

\lstset{basicstyle=\footnotesize\ttfamily}
\tolerance=2048
\lefthyphenmin=2
\hyphenpenalty=128

\begin{document}

\title{A Tactical Communication Protocol Link32 \\
       and Skynet Reference Implementation \\
       for Affordable LTE 4G-based \\
       Drone Swarm Coordination \\
       in Cluster Hybrid Topologies}

\author{Namdak Tonpa}
\date{June 11, 2025}
\maketitle

\begin{abstract}
Coordinating large-scale drone swarms in battlefield or demonstration scenarios
is a complex challenge, particularly under constraints of cost, scalability,
and communication reliability. This article addresses the problem of designing an affordable,
LTE 4G-based swarm coordination framework for micro air vehicles (MAVs) using cluster hybrid
topologies, with low-resolution video streams from cluster lead drones for
command-and-control (C2C) monitoring. We propose a hybrid architecture combining
terrestrial and UAV-based base stations, leveraging low-bandwidth
payloads (100 bytes at 10 Hz) and minimal video streaming (200 kbps per cluster lead).
Key algorithms for path planning, collision avoidance, task allocation, formation control,
and communication optimization are reviewed, drawing from bio-inspired and AI-driven methods.
The solution supports approximately 5,000 drones over a 30 km radius, balancing cost,
scalability, and performance.

Link32 is a tactical communication protocol designed for low-latency, secure, and scalable data
exchange in contested environments, enabling swarm drone coordination, real-time position location
information (PLI), command and control (C2), and tactical chat over UDP-based multicast networks.
Skynet, its reference implementation, is a lightweight C99 framework inspired by LTE’s QoS Class
Identifier (QCI) framework, tailored for military applications and resource-constrained devices
such as drones and embedded systems. With minimal dependencies and a focus on portability, Skynet
supports dynamic TDMA slot management, AES-256-GCM encryption, and lock-free concurrency, ensuring
robust performance in mission-critical scenarios.
\end{abstract}

\newpage
\tableofcontents

\newpage
\section{Introduction}
Link32 is a tactical real-time communication protocol inspired by standards such as
VMF, TSM, SRW, TDL \cite{tdl} (message formats) along with MQTT \cite{mqtt5},
LTE (HSS, PDCP, MME, QCI), Link16 \cite{link16std} (telecommunication protocols),
tailored for military applications requiring robust, low-latency, and secure data
exchange in contested environments. It facilitates swarm drone coordination,
real-time position location information (PLI), command and control (C2),
and tactical chat using UDP-based multicast networks.

Skynet is the reference server implementation of Link32, developed in C99 with minimal dependencies
to ensure portability and performance on resource-constrained devices. Its convergence layer draws
inspiration from LTE’s QoS Class Identifier QCI framework \cite{3gpp}, enabling granular QoS control and
dynamic resource allocation for mission-critical communications.

% Introducing the context and importance of drone swarm coordination
The proliferation of micro air vehicles (MAVs) has enabled applications
such as battlefield surveillance, search and rescue, and large-scale demonstrations.
Coordinating thousands of drones requires robust architectures and algorithms to
manage communication, navigation, and task allocation in dynamic environments.
Traditional centralized systems face scalability issues, while high-cost 5G
solutions are impractical for affordable deployments. This article proposes
an affordable, LTE 4G-based swarm coordination framework using cluster hybrid
topologies, where each drone transmits a 100-byte payload (position, velocity,
tactical data) at 10 Hz, and 5--10 cluster lead drones stream low-resolution
video (200 kbps) for C2C monitoring. We focus on a 30 km radius operational
area, addressing the challenges of bandwidth, latency, and cost through
hybrid architectures and optimized algorithms.

\section{Problem Statement}
% Defining the specific problem of affordable swarm coordination
The problem is to design a cost-effective LTE 4G-based communication and coordination system for a swarm of approximately 5,000 MAVs operating over a 30 km radius in a battlefield or demonstration setting. Each drone sends a 100-byte payload (position, velocity, tactical data) at 10 Hz, requiring approximately 9.6 kbps per drone, while 5--10 cluster lead drones stream low-resolution video at 200 kbps each for C2C monitoring. The total uplink bandwidth is approximately 50 Mbps. The system must:
\begin{itemize}
    \item Ensure reliable communication within LTE 4G constraints (50 Mbps uplink, 20--100 ms latency).
    \item Achieve 30 km coverage using minimal, low-cost equipment.
    \item Support scalable coordination for path planning, collision avoidance, task allocation, and formation control.
    \item Maintain affordability through open-source or commercial off-the-shelf (COTS) components.
\end{itemize}
Challenges include limited bandwidth, latency constraints, signal interference in battlefield environments, and the need for decentralized control to scale to thousands of drones.

\section{Cluster Hybrid Topology}
% Describing the hybrid architecture for swarm coordination
To address the problem, we propose a cluster hybrid topology combining a terrestrial LTE 4G base station with one or more UAV-based aerial base stations (ABSs). The architecture is designed for affordability and scalability, leveraging open-source solutions and COTS hardware.

\subsection{Terrestrial Base Station}
% Outlining the terrestrial component
A single terrestrial base station, implemented using srsRAN with a LimeSDR Mini (cost: $\sim$\$1,500), provides core coverage of 10--15 km. A high-gain omnidirectional antenna (e.g., Laird FG24008, $\sim$\$150) and a 20 W RF amplifier ($\sim$\$500) extend the range toward 20 km. The base station connects to an open-source Evolved Packet Core (EPC) like Open5GS on a low-cost server ($\sim$\$500), handling up to 2,000 simultaneous connections.

\subsection{UAV-Based Aerial Base Station}
% Describing the UAV relay component
One custom-built UAV (cost: $\sim$\$2,000) equipped with a LimeSDR Mini ($\sim$\$300) and a lightweight antenna (e.g., Taoglas GW.26, $\sim$\$50) serves as an aerial base station at 100--120 m altitude, extending coverage to 20--30 km via line-of-sight (LoS) propagation. The UAV rotates with spares to ensure continuous operation, supported by a portable generator ($\sim$\$1,100) and battery swap system ($\sim$\$300).

\subsection{Cluster-Based Swarm Organization}
% Explaining the clustering approach
The swarm is organized into clusters of 50--100 drones, each led by a cluster head responsible for intra-cluster coordination and LTE communication with the base station. Cluster heads transmit 100-byte payloads (9.6 kbps) and, for 5--10 designated leaders, low-resolution video streams (200 kbps). Drone-to-drone (D2D) communication via LTE sidelink reduces network load, enabling scalability to 5,000 drones within the 50 Mbps uplink capacity.

\subsection{Backhaul and Ground Control}
% Detailing connectivity and control
A point-to-point microwave link (e.g., Ubiquiti airFiber 24, $\sim$\$1,500) or satellite terminal (e.g., Starlink, $\sim$\$600) provides backhaul to a command center. A rugged laptop with open-source software like QGroundControl ($\sim$\$500) manages UAV flight and network configuration. Total cost is approximately \$8,400, making the system affordable for battlefield or demonstration use.

\section{Swarm Coordination}
% Summarizing algorithms for coordination tasks
The coordination of 5,000 MAVs requires algorithms for path planning, collision avoidance, task allocation, formation control, and communication optimization. These algorithms are optimized for low computational overhead and LTE 4G constraints, drawing from bio-inspired and AI-driven approaches. Below, we describe each algorithm, highlighting its essence and relevance to the proposed architecture.

\subsection{Path Planning and Navigation}
% Introducing path planning algorithms
Effective path planning ensures drones navigate efficiently in complex environments, balancing computational simplicity with adaptability to dynamic conditions.

\subsubsection{A* Algorithm}
% Describing the essence of A*
The A* algorithm is a cornerstone of path planning, leveraging a graph-based approach to find the shortest path from a drone’s current position to its target. By combining heuristic estimates with actual costs, A* ensures optimal trajectories while integrating with genetic algorithms for real-time optimization in cluttered environments, such as battlefields with obstacles. Its computational efficiency suits micro drones with limited processing power \cite{Hart1968, Liu2019}.

\subsubsection{Particle Swarm Optimization (PSO)}
% Highlighting PSO’s flocking-inspired approach
Inspired by the flocking behavior of birds, PSO optimizes drone trajectories by treating each drone as a particle exploring a solution space. Particles adjust their paths based on local (individual best) and global (swarm best) solutions, making PSO ideal for target tracking and search missions in dynamic settings. Its distributed nature aligns with cluster-based topologies \cite{Kennedy1995, Zhang2017}.

\subsubsection{Ant Colony Optimization (ACO)}
% Capturing ACO’s pheromone-based routing
ACO mimics the pheromone trails of ants to discover optimal paths in Flying Ad Hoc Networks (FANETs). Drones share virtual pheromone data to guide routing and task allocation, enabling efficient navigation in large-scale swarms. Its robustness to dynamic changes suits battlefield scenarios with intermittent connectivity \cite{Dorigo1997, Yang2018}.

\subsubsection{Differential Evolution}
% Explaining adaptive parameter tuning
Differential Evolution dynamically adjusts swarm coordination parameters, such as trajectory weights, by evolving a population of candidate solutions. This adaptability ensures drones respond to environmental changes, optimizing paths in real-time for tasks like formation control. Its lightweight computation is suitable for micro drones \cite{Storn1997, Das2016}.

\subsubsection{Kalman Filter}
% Detailing position estimation in uncertainty
The Kalman Filter estimates drone positions and velocities by fusing noisy sensor data, critical for navigation in GPS-denied environments like urban canyons or forests. Its recursive nature minimizes computational load, enabling robust localization in large swarms \cite{Kalman1960, Welch2006}.

\subsubsection{Spatial-Temporal Joint Optimization}
% Describing synchronized trajectory planning
This approach synchronizes trajectory shapes and timing to optimize navigation in cluttered environments. By jointly considering spatial paths and temporal constraints, it ensures efficient coordination for tasks like surveillance, minimizing energy use and collisions \cite{Lin2018, Chen2020}.

\subsection{Collision Avoidance}
% Introducing collision avoidance strategies
Collision avoidance is critical for dense swarms, ensuring safe navigation without excessive computational or communication demands.

\subsubsection{Artificial Potential Field (APF)}
% Capturing APF’s intuitive force-based avoidance
APF treats drones as particles in a potential field, repelled by obstacles and neighbors while attracted to targets. This intuitive method generates smooth trajectories with low computational cost, though it risks local minima where drones become static. It suits resource-constrained MAVs \cite{Khatib1986, Zhang2020}.

\subsubsection{Reactive Collision Avoidance}
% Highlighting sensor-driven reactivity
Using onboard sensors like time-of-flight, reactive collision avoidance enables drones to detect and avoid obstacles in real-time. Its simplicity and low power requirements make it ideal for micro drones in dynamic battlefield environments \cite{Gageik2015, Lin2020}.

\subsubsection{Flocking Algorithms}
% Describing emergent cohesive movement
Based on Reynolds’ rules (cohesion, separation, alignment), flocking algorithms ensure drones maintain safe distances while moving cohesively. This bio-inspired approach scales well for large swarms, supporting cluster-based coordination \cite{Reynolds1987, OlfatiSaber2006}.

\subsubsection{Vision-Based Avoidance}
% Explaining camera-based detection
Vision-based avoidance uses cameras to detect neighbors and obstacles, offering precise spatial awareness. Despite limitations from field-of-view and lighting, it enhances safety in dense swarms when paired with lightweight sensors \cite{Ross2018, Wang2020}.

\subsection{Task Allocation and Coordination}
% Outlining task allocation methods
Task allocation ensures drones efficiently distribute mission responsibilities, adapting to dynamic conditions with minimal communication.

\subsubsection{Deep Reinforcement Learning (DRL)}
% Capturing DRL’s adaptive learning
DRL, using algorithms like Proximal Policy Optimization (PPO) and Deep Q-Network (DQN), enables drones to learn optimal task allocation and path planning through environmental interaction. Its adaptability excels in dynamic settings like battlefield patrolling \cite{Mnih2015, Schulman2017}.

\subsubsection{Stigmergy}
% Highlighting indirect coordination
Inspired by social insects, stigmergy allows drones to leave virtual ``pheromones'' (data markers) to guide others toward tasks, reducing communication overhead. This decentralized approach suits LTE-constrained swarms \cite{Theraulaz1999, Beckers2000}.

\subsubsection{Consensus Algorithms}
% Describing agreement in uncertainty
Consensus algorithms ensure drones agree on shared goals or states, even under poor LTE connectivity. They are critical for merging swarms or maintaining coordination in disrupted environments \cite{OlfatiSaber2007, Ren2007}.

\subsubsection{Virtual Navigator Model}
% Explaining dynamic path adjustment
This model dynamically adjusts patrol paths based on environmental changes, enhancing flexibility for tasks like surveillance. It integrates with cluster-based topologies for scalable coordination \cite{Low2019, Zhang2022}.

\subsection{Formation Control}
% Explaining formation control techniques
Formation control maintains swarm geometry, balancing precision with robustness to disruptions.

\subsubsection{Leader-Follower}
% Capturing hierarchical guidance
In leader-follower formations, a cluster head guides followers, maintaining relative positions. Its simplicity suits small clusters but is vulnerable to leader failure \cite{Consolini2008, Wang2019}.

\subsubsection{Virtual Structure}
% Describing rigid formation control
The virtual structure approach treats the swarm as a rigid geometric shape, assigning each drone a fixed position. It ensures precise formations but lacks flexibility in dynamic environments \cite{Lewis1997, Ren2008}.

\subsubsection{Behavior-Based}
% Highlighting emergent scalability
Behavior-based methods use simple rules (e.g., avoid collisions, stay near neighbors) to achieve emergent formations. Their scalability makes them ideal for large swarms \cite{Balch1998, Reynolds2000}.

\subsubsection{Consensus-Based}
% Explaining robust formation maintenance
Consensus-based methods share state information to maintain formations, robust to communication disruptions. They suit LTE 4G environments with variable connectivity \cite{Ren2007b, Dong2016}.

\subsubsection{Graph-Based Models}
% Detailing flexible topology control
Graph-based models represent drones as vertices and communication links as edges, enabling flexible formation adjustments. They support dynamic reconfiguration in cluster-based swarms \cite{Zavlanos2007, Mesbahi2010}.

\subsection{Communication Optimization}
% Detailing communication algorithms
Communication optimization ensures efficient data exchange within LTE 4G constraints, minimizing latency and energy use.

\subsubsection{Reactive-Greedy-Reactive (RGR) Protocol}
% Describing hybrid routing efficiency
RGR combines Ad Hoc On-Demand Distance Vector (AODV) with Greedy Geographic Forwarding (GGF) to reduce latency in FANETs. It adapts to dynamic topologies, ensuring reliable packet delivery \cite{Li2016, Zhang2018b}.

\subsubsection{Bee Colony Optimization (BCO)}
% Highlighting bio-inspired routing
BCO mimics bee foraging to optimize routing in FANETs, balancing exploration and exploitation for efficient communication in large swarms \cite{Bitencourt2016, Karaboga2012}.

\subsubsection{EPOS}
% Explaining energy-aware optimization
EPOS is a decentralized multi-agent learning algorithm that optimizes energy consumption and task allocation for spatio-temporal sensing, ideal for LTE-constrained swarms \cite{Chen2019, Liu2021}.

\subsubsection{Graph Attention-Based Decentralized Actor-Critic}
% Capturing dual-objective control
This method uses graph neural networks to enable dual-objective control (task completion, energy efficiency), enhancing communication efficiency in cluster-based topologies \cite{Zhang2020, Wang2021}.


\subsection{Principles}
Link32 adheres to the following design principles:
\begin{itemize}
    \item \textbf{Threat Model}: Confidentiality and integrity, no non-repudiation.
    \item \textbf{Latency}: Microsecond-precision timings .
    \item \textbf{Implementation}: Written in C99 for portability and performance.
    \item \textbf{Key Provisioning}: Key distribution for controlled setup.
    \item \textbf{Mandatory Encryption}: All messages encrypted with AES-256-GCM.
    \item \textbf{Node Identification}: Node names hashed to 32-bit using FNV-1a.
    \item \textbf{Lock-Free Design}: Uses atomic compare-and-swap for concurrency.
    \item \textbf{Topic Architecture}: Topics map to IP multicast groups.
    \item \textbf{Queue Management}: Global network queue, per-topic subscribes.
    \item \textbf{Key Storage}: Separate key stores per executable.
\end{itemize}

\newpage

\subsection{Properties}
Link32 and Skynet are designed with the following properties:
\begin{itemize}
    \item \textbf{Message Size}: 32-byte header (48 bytes payload with GCM tag).
    \item \textbf{Security}: ECDH key exchange SECP384R1, Mandatory AES-256-GCM.
    \item \textbf{Non-blocking}: Using monotonic clocks and non-blocking I/O.
    \item \textbf{Footprint}: $\sim$64KB L1 cache usage, $\sim$2000 lines of code (LOC).
    \item \textbf{Swarm Scalability}: Supports large-scale drone swarms.
    \item \textbf{Dependencies}: Single dependency on OpenSSL for cryptography.
\end{itemize}

\section{Link32 Protocol}

\subsection{S-Message Format}
The Link32 message structure, \texttt{SkyNetMessage}, is compact for large-scale swarm
communication:
\begin{lstlisting}
typedef struct {
  uint8_t version : 4;    // Protocol version (current: 1)
  uint8_t type : 4;       // Message type (0-6)
  uint8_t qos : 4;        // Quality of Service (0-3)
  uint8_t hop_count : 4;  // Hop count for routing (0-15)
  uint32_t npg_id;        // Topic identifier (1-103)
  uint32_t node_id;       // Sender node ID (FNV-1a hash)
  uint32_t seq_no;        // Sequence number for deduplication
  uint8_t iv[16];         // AES-256-GCM initialization vector
  uint16_t payload_len;   // Payload length (0-32767)
  uint8_t payload[MAX_BUFFER];     // GCM-TAG
} SkyNetMessage;
\end{lstlisting}
\begin{itemize}
    \item \textbf{Header Size}: 32 bytes.
    \item \textbf{Minimum Payload Size}: 48 bytes minimum (32-byte + GCM tag).
    \item \textbf{Maximum Payload Size}: Up to 32720 bytes.
\end{itemize}

\newpage
\subsection{Message Types}
The protocol defines seven message types, as shown in Table 1.

\begin{table}[h]
\centering
\caption{Link32 Message Types}
\begin{tabular}{clp{6cm}}
\hline
\textbf{Type} & \textbf{Name} & \textbf{Description} \\
\hline
0 & Key Exchange & Exchanges ECC public keys for ECDH session setup. \\
1 & Slot Request & Requests a TDMA slot from the server. \\
2 & Chat         & Sends tactical chat messages. \\
3 & Ack          & Acknowledges slot assignments or control messages. \\
4 & Waypoint     & Specifies navigation waypoints for C2. \\
5 & Status       & Reports position, velocity, or sensor data (e.g., PLI). \\
6 & Formation    & Coordinates swarm formations. \\
\hline
\end{tabular}
\end{table}

\subsection{Multicast Topics}
Link32 uses multicast topics mapped to IP multicast groups, as in Table 2.

\begin{table}[h]
\centering
\caption{Multicast Topics}
\begin{tabular}{cllp{5cm}}
\hline
\textbf{NPG} & \textbf{Name} & \textbf{Multicast} & \textbf{Purpose} \\
\hline
1   & npg\_control      & 239.255.0.1   & Kkey exchange, Slot requests. \\
6   & npg\_pli          & 239.255.0.6   & Position information (status). \\
7   & npg\_surveillance & 239.255.0.7   & Sensor data forwarding. \\
29  & npg\_chat         & 239.255.0.29  & Tactical chat and acks. \\
100 & npg\_c2           & 239.255.0.100 & C2 (waypoints, formations). \\
101 & npg\_alerts       & 239.255.0.101 & Network alerts and self-healing. \\
102 & npg\_logistics    & 239.255.0.102 & Logistical coordination. \\
103 & npg\_coord        & 239.255.0.103 & Swarm Coordination. \\
\hline
\end{tabular}
\end{table}

\newpage
\subsection{Slot Management}
Link32 employs a Time Division Multiple Access (TDMA)-like slot manager to minimize collisions:
\begin{itemize}
    \item \textbf{Slot Array}: Fixed-size (256) array \texttt{slots}.
    \item \textbf{Dynamic Topics}: Each slot creates a temporary multicast group.
    \item \textbf{Allocation}: First-come, first-serve with no timeouts.
    \item \textbf{Timing}: Slots cycle every \texttt{TIME\_SLOT\_INTERVAL\_US=1000$\mu$s}.
\end{itemize}
Clients send \texttt{SKYNET\_MSG\_SLOT\_REQUEST} to NPG 1, and the server assigns slots via
\texttt{SKYNET\_MSG\_ACK}.

\subsection{Deduplication}
A fixed-size circular buffer (\texttt{seq\_cache}) prevents message loops:
\begin{itemize}
    \item \textbf{Structure}: Stores \texttt{node\_id}, \texttt{seq\_no}.
    \item \textbf{Memory}: $\sim$16KB (1024 $\times$ 16 bytes).
    \item \textbf{Complexity}: O(1) lookup using FNV-1a hashing.
    \item \textbf{Threshold}: Discards duplicates within 1 second.
\end{itemize}

\subsection{Security}
Security mechanisms include:
\begin{itemize}
    \item \textbf{Key Exchange}: ECDH over secp384r1 for 256-bit AES keys.
    \item \textbf{Encryption}: AES-256-GCM with 16-byte IV and 16-byte tag.
    \item \textbf{Key Storage}: Per Process in \texttt{ec\_priv} and \texttt{ec\_pub}.
    \item \textbf{Key Derivation}: HKDF-SHA256 for AES keys.
    \item \textbf{Self-messages}: Skips (drops)  messages.
\end{itemize}

\newpage
\subsection{Subscriptions}
Nodes subscribe to topics based on roles, as shown in Table 3:
\begin{table}[h]
\centering
\caption{Role-Based Subscriptions}
\begin{tabular}{lllp{5cm}}
\hline
\textbf{Role} & \textbf{NPGs} & \textbf{Purpose} \\
\hline
Infantry   & 1, 29                  & Network control and tactical chat. \\
Drone      & 1, 6, 7, 100, 101      & C2, PLI, surveillance, alerts. \\
Air        & 1, 6, 7, 100, 101, 103 & C2, PLI, surveillance, alerts, coord. \\
Sea        & 1, 7, 29, 102, 103     & C2, surveillance, chat, logistics, coord. \\
Ground     & 1, 7, 29, 102          & C2, surveillance, chat, logistics. \\
Relay      & 1, 6, 101              & C2, PLI, alerts for relaying. \\
Controller & 1, 6, 100, 101         & C2, PLI, alerts for command posts. \\
\hline
\end{tabular}
\end{table}

\section{Skynet Implementation}

\subsection{Dependencies}
\begin{itemize}
    \item \textbf{OpenSSL}: For ECC, ECDH, and AES-256-GCM.
    \item \textbf{C99 Compiler}: GCC or equivalent.
    \item \textbf{POSIX Environment}: For threading, epoll, and timerfd.
\end{itemize}

\subsection{Build}
To build Skynet:
\begin{lstlisting}
# git clone git@github.com:BitEdits/skynet
# cd skynet
# cc -o skynet_client skynet_client.c skynet_proto.c -lcrypto
# cc -o skynet skynet.c skynet_proto.c skynet_conv.c -lcrypto
\end{lstlisting}

\newpage
\subsection{Installation}
The provisioning script \texttt{skynet.sh} generates ECC key pairs:
\begin{lstlisting}
# ./skynet.sh
Generated keys for node npg_control (hash: 06c5bc52) in /secp/
Generated keys for node npg_pli (hash: c9aef284) in /secp/
Generated keys for node npg_surveillance (hash: 4d128cdc) in /secp/
Generated keys for node npg_chat (hash: 9c69a767) in /secp/
Generated keys for node npg_c2 (hash: 89f28794) in /secp/
Generated keys for node npg_alerts (hash: 9f456bca) in /secp/
Generated keys for node npg_logistics (hash: 542105cc) in /secp/
Generated keys for node npg_coord (hash: e46c0c22) in /secp/
Generated keys for node server (hash: 40ac3dd2) in /secp/
Generated keys for node client (hash: 8f929c1e) in /client/secp/
# cp /secp/*.ec_pub /client/secp/
\end{lstlisting}

\subsection{Server Operation}
The server binds to UDP port 6566, joins multicast groups, and processes messages using a global
queue. Example output:
\begin{lstlisting}
# skynet server
Node 40ac3dd2 bound to 0.0.0.0:6566.
Joined multicast group 239.255.0.1 (NPG 1: control).
Joined multicast group 239.255.0.6 (NPG 6: PLI).
Message received, from=8f929c1e, to=1, size=231.
Decryption successful, from=8f929c1e, to=1, size=215.
Saved public key for client 8f929c1e.
Assigned slot 0 to node 8f929c1e.
Message received, from=8f929c1e, to=6, size=40.
Decryption successful, from=8f929c1e, to=6, size=24.
Message sent from=8f929c1e, to=6, seq=3, multicast=239.255.1.0.
\end{lstlisting}

\subsection{Client Operation}
The client joins topic-specific multicast groups and sends key exchange, slot requests, and status
messages. Example output:
\begin{lstlisting}
# skynet_client client
Node 8f929c1e connecting to port 6566.
Joined multicast group 239.255.0.1 (NPG 1).
Joined multicast group 239.255.0.6 (NPG 6).
Sent key exchange message to server.
Sent slot request message to server.
Received slot assignment: slot=0.
Joined slot group 2399540.1.
Sent status message: multicast=239.255.1.0,
                     pos=[0.1, 0.1, 0.1],
                     vel=[0.0, 0.0, 0.0],
                     seq=2.
\end{lstlisting}

\newpage
\subsection{Usage}
Skynet includes five utilities:

\subsubsection*{Keys Provisioning}
Generates ECC key pairs.
\begin{lstlisting}
skynet_keygen <node> [--server|--client]
\end{lstlisting}

\subsubsection*{Message Encryption}
Encrypts a test message to \texttt{<npg\_id>.sky}.
\begin{lstlisting}
skynet_encrypt <sender> <recipient> <file>
\end{lstlisting}

\subsubsection*{Message Decryption}
Decrypts \texttt{<file.sky>}.
\begin{lstlisting}
skynet_decoder <sender> <recipient> <file.sky>
\end{lstlisting}

\subsubsection*{Skynet Server}
Runs the server with FNV-1a hashed \texttt{<node>}.
\begin{lstlisting}
skynet <node>
\end{lstlisting}

\subsubsection*{Skynet Client}
Runs the client with FNV-1a hashed \texttt{<node>}.
\begin{lstlisting}
skynet_client <node>
\end{lstlisting}

\subsection{Limitations}
\begin{itemize}
    \item \textbf{Slot Scalability}: Fixed \texttt{SLOT\_COUNT=256} limits nodes to 256.
    \item \textbf{No Retransmission}: Dropped messages are not retransmitted.
    \item \textbf{Key Management}: Manual public key copying required.
    \item \textbf{Deduplication}: \texttt{SEQ\_CACHE\_SIZE=1024} may cause collisions.
\end{itemize}

\section{Convergence Architecture}
The Skynet system's convergence layer employs a 3-level structural hierarchy comprising Quality of
Service (QoS), Bearer, and Entity components. This design is driven by the need for granular QoS
control, per-node resource management, reliable packet delivery, dynamic slot allocation, and
scalability in a Time Division Multiple Access (TDMA)-based tactical network. Inspired by the
Long-Term Evolution (LTE) QoS Class Identifier (QCI) framework, the hierarchy ensures military-grade
performance, including latency below 50ms for command-and-control (C2) traffic and reliable delivery
for critical communications, such as swarm drone coordination. Compared to alternative designs, such
as flat or centralized structures, this approach excels in dynamic, resilient scenarios, providing
robust and scalable QoS management.

\subsection{Flat QoS Structure}
A flat QoS structure assigns slots directly to Network Protocol Groups (NPGs) based on their QoS
profiles, without intermediate bearer or entity layers. For example, an NPG like
\texttt{SKYNET\_NPG\_C2} might be statically allocated three slots with QoS level 3. While simpler
and requiring less memory, this approach lacks per-node isolation, making it unsuitable for dynamic
networks with multiple nodes sharing the same NPG. It also complicates reliable delivery, as there is
no mechanism for per-flow reordering or sequence tracking. Additionally, static slot assignments
cannot adapt to node arrivals or departures, leading to inefficient resource utilization. The 3-level
hierarchy overcomes these limitations by introducing bearers for flow isolation and entities for
node-level coordination, enabling dynamic and scalable QoS management.

\begin{lstlisting}
typedef struct {
    uint32_t npg_id;
    uint8_t qos;
    uint32_t slot_count;
    uint32_t slot_ids[MAX_QOS_SLOTS];
    uint8_t priority;
} QoSSlotAssignment;
\end{lstlisting}

\subsection{Hierarchical QoS Structure}
The 3-level hierarchy separates concerns into QoS, Bearer, and Entity layers, each addressing
specific aspects of network management. The \texttt{SkyNetBearerQoS} structure defines QoS
parameters such as priority (1--15, where 1 is highest), delay budget (in milliseconds), reliability
(best-effort or reliable), and minimum TDMA slots, allowing precise service differentiation. The
\texttt{SkyNetBearer} represents a logical communication channel for a node-NPG pair, encapsulating
QoS parameters, assigned slots, and state for reordering and reliability. The
\texttt{SkyNetConvergenceEntity} aggregates bearers for a single node, managing slot requests and
coordinating resource allocation. This modular design ensures scalability, flexibility, and
robustness, outperforming flat structures by providing flow isolation and dynamic adaptation.

\begin{lstlisting}
typedef struct {
    uint8_t priority;        // 1-15 (1 = highest)
    uint32_t delay_budget_ms;// Delay tolerance (ms)
    uint8_t reliability;     // 0 (best-effort), 1 (reliable)
    uint32_t min_slots;      // Minimum TDMA slots
} SkyNetBearerQoS;
\end{lstlisting}

\begin{lstlisting}
typedef struct {
    uint32_t bearer_id;      // Unique bearer ID
    SkyNetBearerQoS qos;     // QoS parameters
    uint32_t node_id;        // Owning node
    uint32_t npg_id;         // Associated NPG
    uint32_t assigned_slots[SKYNET_MAX_SLOTS]; // Assigned slot IDs
    uint32_t slot_count;     // Number of assigned slots
    SkyNetMessage reorder_queue[SKYNET_REORDER_SIZE];
    uint32_t expected_seq_no;// Next expected sequence number
    uint32_t last_delivered; // Last delivered sequence number
    uint64_t last_reorder_time_us; // Last reorder check
} SkyNetBearer;
\end{lstlisting}

\begin{lstlisting}
typedef struct {
    SkyNetBearer bearers[SKYNET_MAX_BEARERS]; // Active bearers
    uint32_t bearer_count;   // Number of active bearers
    atomic_uint slot_requests_pending; // Pending slot requests
} SkyNetConvergenceEntity;
\end{lstlisting}

\subsection{Granular QoS Control}
Tactical networks handle diverse traffic types, such as C2, position location information (PLI), and
chat, each with distinct latency, reliability, and bandwidth requirements. A uniform QoS approach
fails to meet these needs. The \texttt{SkyNetBearerQoS} structure enables granular control by
defining specific parameters for each bearer. For instance, \texttt{SKYNET\_NPG\_CONTROL} (NPG 1,
QoS 3) is assigned three slots with a low delay budget to ensure timely C2 delivery, while
\texttt{SKYNET\_NPG\_CHAT} (NPG 103, QoS 0) receives one slot for best-effort traffic. This
differentiation, inspired by LTE QCI, guarantees that high-priority traffic meets stringent military
requirements, such as sub-50ms latency for C2, while optimizing resource allocation for
lower-priority flows.

\subsection{Per-Node and Per-Flow Resource Management}
Each node in the network may support multiple concurrent flows (e.g., C2, PLI, control) with varying
QoS needs. Without isolation, these flows compete for resources, risking contention and degraded
performance. The \texttt{SkyNetBearer} structure provides flow-level isolation by associating each
bearer with a specific node-NPG pair, tracking its assigned slots and QoS parameters. The
\texttt{SkyNetConvergenceEntity} groups all bearers for a node, enabling centralized resource
management and preventing one flow from starving others. This per-node and per-flow approach ensures
efficient slot allocation, supports multiple simultaneous communications, and scales to accommodate
dynamic network topologies.

\subsection{Reliability and Reordering}
TDMA networks may deliver packets out of order due to slot scheduling or retransmissions,
particularly for reliable traffic (e.g., \texttt{reliability=1}). The bearer includes
a \texttt{reorder\_queue} and tracks \texttt{expected\_seq\_no} and \texttt{last\_delivered} to
reorder packets and ensure reliable delivery. The entity coordinates
reorder checks across bearers using \texttt{last\_reorder\_time\_us}, minimizing overhead. This
mechanism is critical for applications requiring guaranteed delivery, such as C2 or control
messages, and enhances robustness compared to flat structures, which lack per-flow reordering
capabilities.

\subsection{Dynamic Slot Allocation}
Tactical networks operate in dynamic environments where nodes join or leave, and traffic patterns
shift. Static slot assignments are inefficient and inflexible. The entity
tracks \texttt{slot\_requests\_pending} and manages bearer slot assignments via
\texttt{assigned\_slots} and \texttt{slot\_count}. The bearer parameter
\texttt{min\_slots} ensures minimum resource guarantees, while the entity facilitates dynamic
re-allocation through the \texttt{skynet\_convergence\_schedule\_slots} function. Weighted Fair
Queuing, driven by bearer priorities, optimizes slot distribution, ensuring high-QoS traffic
receives preferential treatment. This adaptability is a key advantage over static or centralized
designs.

\subsection{Scalability and Modularity}
A flat QoS structure becomes unwieldy as the number of nodes and NPGs increases, complicating
scheduling and state management. The 3-level hierarchy addresses this by separating concerns: QoS
defines service requirements, bearers manage flow-specific state, and entities coordinate node-level
convergence. This modular design scales to support large networks, simplifies debugging, and
facilitates maintenance. The entity layer aggregates bearer state, reducing scheduling complexity
from O($n$) for $n$ bearers to O($m$) for $m$ nodes. The hierarchy also supports future
enhancements, such as preemption or adaptive QoS, without requiring a system overhaul.

\subsection{LTE QCI Compatibility}
The 3-level hierarchy draws inspiration from LTE's QCI framework, which uses bearers with QoS
profiles to manage diverse traffic types (e.g., VoIP, video, best-effort) per User Equipment (UE).
By adapting this model to TDMA, Skynet replaces LTE's EPS bearers with TDMA slot-based bearers,
leveraging telecom best practices. The bearer parameters mirror QCI attributes,
such as priority and delay budget, ensuring compatibility with established standards. This alignment
reduces design risk, enhances interoperability with telecom systems, and provides a familiar
framework for engineers, making it easier to develop and maintain the system.

\subsection{MQTT Compatibility}
Skynet’s convergence layer maps its QoS levels to MQTT’s QoS semantics (ISO/IEC 20922:2016) and LTE’s
QCI framework (3GPP TS 23.203, Release 15), enabling compatibility with both standards for tactical
communications. MQTT defines three QoS levels:
\begin{itemize}
    \item \textbf{QoS 0 (At Most Once)}: Best-effort delivery with no acknowledgment or retransmission.
          Suitable for non-critical data like tactical chat where packet loss is tolerable.
    \item \textbf{QoS 1 (At Least Once)}: Guarantees at least one delivery with acknowledgment (ACK),
          allowing duplicates. Used for data requiring delivery, such as position location information
          (PLI), where duplicates are acceptable.
    \item \textbf{QoS 2 (Exactly Once)}: Ensures exactly one delivery via a four-way handshake
          (\texttt{PUBLISH}, \texttt{PUBREC}, \texttt{PUBREL}, \texttt{PUBCOMP}). Critical for
          command and control (C2) messages requiring no loss or duplication.
\end{itemize}
Skynet implements these using its \texttt{SkyNetBearerQoS} and \texttt{SkyNetBearer} structures,
configured via \texttt{skynet\_convergence\_init}. QoS 0 skips \texttt{reorder\_queue} and ACKs, QoS 1
uses ACKs (\texttt{SKYNET\_MSG\_ACK}) with retransmission on timeout (100ms), and QoS 2 employs a
handshake with strict ordering and deduplication via \texttt{seq\_no}. All Network Protocol Groups
(NPGs) and message types are mapped to QoS levels, as shown in Table 4.

\begin{table}[h]
\centering
\caption{Skynet QoS Mapping to MQTT and LTE QCI}
\begin{tabular}{cllllll}
\hline
\textbf{QoS} & \textbf{Priority} & \textbf{NPG} & \textbf{QCI} & \textbf{Budget} & \textbf{Slots} & \textbf{Type} \\
\hline
0 & 15 & 29  & QCI 9 & 300ms & 1 & 2 \\
0 & 14 & 102 & QCI 9 & 300ms & 1 & 2, 5 \\
0 & 13 & 103 & QCI 9 & 300ms & 1 & 2, 4 \\
1 & 7  & 6   & QCI 7 & 100ms & 2 & 5 \\
1 & 6  & 7   & QCI 7 & 100ms & 2 & 5 \\
2 & 3  & 1   & QCI 5 & 50ms  & 3 & 0, 1, 3 \\
2 & 2  & 100 & QCI 3 & 50ms  & 3 & 3, 4, 6 \\
2 & 1  & 101 & QCI 5 & 50ms  & 3 & 3, 6 \\
\hline
\end{tabular}
\end{table}

This mapping ensures military-grade performance: QoS 2 meets sub-50ms latency for C2 (\texttt{npg\_c2},
NPG 100) and control (\texttt{npg\_control}, NPG 1), QoS 1 supports reliable PLI (\texttt{npg\_pli},
NPG 6) with 100ms latency, and QoS 0 optimizes bandwidth for chat (\texttt{npg\_chat}, NPG 29). The
\texttt{SkyNetConvergenceEntity} dynamically allocates slots via Weighted Fair Queuing, prioritizing
high-QoS bearers, while \texttt{reorder\_queue} ensures reliability for QoS 1 and 2, aligning with
LTE QCI’s packet error loss rates (e.g., 10$^{-6}$ for QCI 5).

In conclusion, the 3-level hierarchy, inspired by LTE QCI, enables Skynet to replicate MQTT’s QoS
semantics, ensuring scalable, low-latency, and reliable communication for swarm drone coordination and
other tactical applications.

\newpage
\section{Deploying Private LTE 4G Network}
This section provides a concise guide for administrators to deploy a private LTE 4G network
using srsRAN with a USRP B200mini or USRP X310 FPGA-based Software Defined Radios (SRD)
and iPhone User Equipment as UE LTE terminals in battlefield EDGE scenarios.
Private LTE 4G Network acts as a Data Link Layer (Media Access Control) for Skynet,
which operates as Network and Transport Layer framework for drone swarm coordination.
It outlines the main functionality of srsRAN, key configuration files, and a technical
overview of LTE 4G components relevant to the deployment.

\subsection{LTE 4G Overview and srsRAN Components}
LTE 4G is a cellular standard providing high-speed, low-latency communication through
a layered architecture. Key components include:
\begin{itemize}
\item \textbf{Evolved NodeB (eNB)}: Manages radio resources, handles UE connections, and interfaces with the core network. In srsRAN, the eNB is implemented using a USRP X310 for radio transmission/reception.
\item \textbf{Evolved Packet Core (EPC)}: Handles network control, authentication, and IP connectivity. srsRAN uses Open5GS or similar for EPC functions (MME, HSS, SP-GW).
\item \textbf{UE}: Devices (e.g., iPhones) connecting to the eNB. Requires programmable SIMs for private network authentication.
\item \textbf{Protocol Stack}: Includes PHY (physical layer), MAC (medium access control), RLC (radio link control), PDCP (packet data convergence protocol), and RRC (radio resource control) for reliable data transfer and QoS management.
\end{itemize}

srsRAN leverages open-source software and COTS hardware (e.g., USRP X310) to create a
private LTE network, supporting secure, low-latency communication for tactical applications
like drone coordination and C2C.

\subsection{Deployment Steps}
To deploy srsRAN at the battlefield EDGE:
\begin{enumerate}
\item \textbf{Hardware Setup}:
\begin{itemize}
\item Use a USRP X310 with a GPSDO (e.g., Leo Bodnar) for clock synchronization.
\item Attach a directional antenna (e.g., VERT2450) to minimize RF footprint.
\item Connect to a rugged server (e.g., Intel NUC, $\sim$\$500) running Ubuntu 22.04+.
\end{itemize}
\item \textbf{Software Installation}:
\begin{itemize}
\item Install srsRAN: \texttt{git clone \url{https://github.com/srsran/srsRAN_4G}}, then build using instructions.
\item Install dependencies: UHD drivers (\texttt{libuhd-dev}), Open5GS, and OpenSSL.
\item Configure kernel for GTP-U: \texttt{modprobe gtp}.
\end{itemize}
\item \textbf{Network Configuration}:
\begin{itemize}
\item Set up a private IP range (e.g., 172.16.0.0/16) for EPC and eNB.
\item Ensure firewall allows UDP ports (e.g., 2152 for GTP-U, 36412 for S1AP).
\end{itemize}
\item \textbf{SIM Provisioning}:
\begin{itemize}
\item Use programmable USIMs (e.g., sysmoUSIM) with custom IMSI and keys.
\item Update \texttt{user\_db.csv} in \texttt{[hss]} with UE credentials.
\end{itemize}
\item \textbf{Configuration Files}: Customize the following srsRAN configuration files (located in \texttt{/root/.config/srsran/}).
\end{enumerate}

\subsection{Configuration Files}
Below are key configurations for srsRAN, tailored for battlefield EDGE with Skynet integration.
Only essential parameters are shown; defaults suffice for others.

\subsubsection{enb.conf}
Configures the eNB for radio and network settings.
\begin{lstlisting}
[enb]
enb_id = 0x19B
mcc = 001
mnc = 01
mme_addr = 127.0.1.100
gtp_bind_addr = 127.0.1.1
s1c_bind_addr = 127.0.1.1
n_prb = 50

[rf]
dl_earfcn = 3350
tx_gain = 15  % Low power to reduce detectability
rx_gain = 40
device_name = uhd
device_args = type=x300,master_clock_rate=184.32e6,send_frame_size=8000

[enb_files]
sib_config = sib.conf
rr_config = rr.conf
rb_config = rb.conf
\end{lstlisting}

\subsubsection{epc.conf}
Configures the EPC for authentication and IP connectivity.
\begin{lstlisting}
[mme]
mme_code = 0x1a
mme_group = 0x0001
tac = 0x0007
mcc = 001
mnc = 01
mme_bind_addr = 127.0.1.100
apn = srsapn
dns_addr = 8.8.8.8
encryption_algo = EEA2
integrity_algo = EIA2

[hss]
db_file = user_db.csv

[spgw]
gtpu_bind_addr = 127.0.1.100
sgi_if_addr = 172.16.0.1
sgi_if_name = srs_spgw_sgi
\end{lstlisting}

\subsubsection{rb.conf}
Defines QoS for radio bearers, aligned with Skynet’s QoS requirements.
\begin{lstlisting}
qci_config = (
{
  qci = 7;
  pdcp_config
    = { discard_timer = -1;
        pdcp_sn_size = 12; }
  rlc_config
    = { ul_um = { sn_field_length = 10; };
        dl_um = { sn_field_length = 10;
                  t_reordering = 45; }; }
  logical_channel_config
    = { priority = 13;
        prioritized_bit_rate = -1;
        bucket_size_duration = 100;
        log_chan_group = 2; }
},
{
  qci = 9;
  pdcp_config
    = { discard_timer = 150;
        status_report_required = true; }
  rlc_config
    = { ul_am = { t_poll_retx = 120;
                  poll_pdu = 64;
                  poll_byte = 750;
                  max_retx_thresh = 16; };
        dl_am = { t_reordering = 50;
                  t_status_prohibit = 50; }; }
  logical_channel_config
    = { priority = 11;
        prioritized_bit_rate = -1;
        bucket_size_duration = 100;
        log_chan_group = 3; }
}
);
\end{lstlisting}

\subsubsection{rr.conf}
Configures radio resources and cell parameters.
\begin{lstlisting}
cell_list = (
{
  cell_id = 0x01;
  tac = 0x0007;
  pci = 1;
  dl_earfcn = 3350;
  ho_active = false;
}
);
\end{lstlisting}

\subsubsection{sib.conf}
Defines System Information Blocks for network discovery.
\begin{lstlisting}
sib1 = { intra_freq_reselection = "Allowed";
         q_rx_lev_min = -65;
         cell_barred = "NotBarred";
         si_window_length = 20; }
sib2 = { rr_config_common_sib
           = { rach_cnfg = { num_ra_preambles = 52;
                             preamble_init_rx_target_pwr = -104;
                             pwr_ramping_step = 6;
                             ra_resp_win_size = 10;
                             mac_con_res_timer = 64;
                             max_harq_msg3_tx = 4; }; }; }
sib3 = { cell_reselection_common
           = { q_hyst = 2; };
         intra_freq_reselection
           = { q_rx_lev_min = -61;
               p_max = 23;
               s_intra_search = 5; }; }
\end{lstlisting}

\subsubsection{ue.conf}
Configures UE (iPhone) settings for network attachment.
\begin{lstlisting}
[rf]
dl_earfcn = 3350
tx_gain = 80
device_name = uhd
device_args = type=x300,master_clock_rate=184.32e6,send_frame_size=8000

[usim]
mode = soft
algo = milenage
opc = 63BFA50EE6523365FF14C1F45F88737D
k = 00112233445566778899aabbccddeeff
imsi = 001010123456780
imei = 353490069873319

[nas]
apn = srsapn
apn_protocol = ipv4
\end{lstlisting}

\subsection{Operation}
\begin{enumerate}
\item Start EPC: \texttt{srsmbd -c epc.conf}.
\item Start eNB: \texttt{srsenb -c enb.conf}.
\item Configure iPhones with programmable USIMs and attach to the network (MCC=001, MNC=01).
\item Integrate with Skynet:
       Configure \texttt{npg\_c2} (NPG 100) and
                \texttt{npg\_pli} (NPG 6) for drone C2 and PLI,
                             using QoS 2 for low-latency C2 (50 ms)
                               and QoS 1 for PLI (100 ms).
\item Monitor logs (\texttt{enb.log}, \texttt{epc.log}) for connectivity issues.
\end{enumerate}

\subsection{Security and Visibility}
\begin{itemize}
\item \textbf{Encryption}: Use AES-256 (EEA2) and integrity protection (EIA2) in \texttt{epc.conf} to secure communications.
\item \textbf{Low Visibility}: Set \texttt{tx\_gain = 15} and use directional antennas to minimize RF footprint.
\item \textbf{Interference Mitigation}: Select Band 3 (EARFCN 3350) to avoid commercial spectrum.
\end{itemize}

\subsection{Troubleshooting}
\begin{itemize}
\item \textbf{UE Connection Issues}: Verify IMSI in \texttt{user\_db.csv}, ensure \texttt{dl\_earfcn} matches, and check GPSDO synchronization.
\item \textbf{GTP-U Errors}: Ensure \texttt{modprobe gtp} and correct \texttt{gtpu\_bind\_addr}.
\item \textbf{Logs}: Set \texttt{all\_level = debug} in \texttt{enb.conf} and \texttt{epc.conf} for detailed diagnostics.
\end{itemize}

\section{Dedication}
The Skynet is dedicated to Vitalii Karvatskyi.

\section{Conclusion}
Link32 and Skynet provide a robust framework for tactical communication, combining low-latency,
security, and scalability for military applications, including swarm drone coordination. Future
improvements could address slot scalability beyond 256 nodes, automated key distribution to replace
manual provisioning, retransmission mechanisms for dropped messages, and enhanced deduplication to
support larger networks.

The proposed LTE 4G-based cluster hybrid topology provides an affordable,
scalable solution for coordinating 5,000 MAVs over a 30 km radius,
with each drone transmitting a 100-byte payload at 10 Hz and 5--10 cluster
leads streaming 200 kbps video for C2C monitoring. The hybrid architecture,
combining a terrestrial base station and UAV relay, achieves full coverage
at a cost of approximately $\sim$ \$10,000, leveraging open-source srsRAN and COTS
hardware. Key algorithms, including PSO, DRL, and consensus-based methods,
enable efficient path planning, collision avoidance, task allocation,
formation control, and communication optimization. Future work should
explore 5G integration for higher capacity and anti-jamming techniques
for battlefield reliability.

\begin{thebibliography}{10}

% Link32 Protocol and Related Information

\bibitem{tdl} U.S. Department of Defense, ``MIL-STD-6016: Tactical Data Link (TDL) J-Message Standard,'' 2008.
\bibitem{link16std} U.S. Department of Defense, ``Link 16 Network Management and Operations,'' ADA404334, 2003.
\bibitem{3gpp} 3GPP, ``TS 23.203: Policy and Charging Control Architecture (Release 15),'' 2018.
\bibitem{mqtt5} ISO/IETF standard ``MQTT Version 5.0,'', 2019.
\bibitem{link16} Maksym Sokhatsky. A Non-Blocking C Implementation of a Link16 Service-Oriented Architecture: Design and Verification. 2025.
\bibitem{gsm} Maksym Sokhatsky. A Non-Blocking C Implementation of a GSM HSS and PDCP Services. 2025.
\bibitem{tdma} J. Li and Y. Zhang, ``TDMA-Based Scheduling for Tactical Wireless Networks,'' IEEE Transactions on Vehicular Technology, vol. 68, no. 5, pp. 4987--4999, 2019.
\bibitem{drone_swarm} A. Sharma and P. Kumar, ``Communication Protocols for UAV Swarm Coordination: A Survey,'' Journal of Network and Computer Applications, vol. 172, 2020.

% Path Planning and Navigation

\bibitem{Hart1968} Hart, P. E., Nilsson, N. J., and Raphael, B., ``A Formal Basis for the Heuristic Determination of Minimum Cost Paths,'' \textit{IEEE Transactions on Systems Science and Cybernetics}, vol. 4, no. 2, pp. 100--107, 1968.
\bibitem{Liu2019} Liu, J., and Yang, J., ``Path Planning for UAVs Using A* and Genetic Algorithms,'' \textit{Journal of Unmanned Aerial Systems}, vol. 5, no. 3, pp. 45--52, 2019.
\bibitem{Kennedy1995} Kennedy, J., and Eberhart, R., ``Particle Swarm Optimization,'' \textit{Proceedings of ICNN'95 - International Conference on Neural Networks}, vol. 4, pp. 1942--1948, 1995.
\bibitem{Zhang2017} Zhang, Y., and Li, S., ``UAV Path Planning Using PSO in Dynamic Environments,'' \textit{IEEE Transactions on Aerospace and Electronic Systems}, vol. 53, no. 4, pp. 1654--1663, 2017.
\bibitem{Dorigo1997} Dorigo, M., and Gambardella, L. M., ``Ant Colony System: A Cooperative Learning Approach to the Traveling Salesman Problem,'' \textit{IEEE Transactions on Evolutionary Computation}, vol. 1, no. 1, pp. 53--66, 1997.
\bibitem{Yang2018} Yang, Q., and Yoo, S. J., ``Optimal UAV Path Planning Using Ant Colony Optimization in FANETs,'' \textit{Ad Hoc Networks}, vol. 78, pp. 54--63, 2018.
\bibitem{Storn1997} Storn, R., and Price, K., ``Differential Evolution – A Simple and Efficient Heuristic for Global Optimization over Continuous Spaces,'' \textit{Journal of Global Optimization}, vol. 11, no. 4, pp. 341--359, 1997.
\bibitem{Das2016} Das, S., and Suganthan, P. N., ``Differential Evolution: A Survey of the State-of-the-Art,'' \textit{IEEE Transactions on Evolutionary Computation}, vol. 15, no. 1, pp. 4--31, 2016.
\bibitem{Kalman1960} Kalman, R. E., ``A New Approach to Linear Filtering and Prediction Problems,'' \textit{Journal of Basic Engineering}, vol. 82, no. 1, pp. 35--45, 1960.
\bibitem{Welch2006} Welch, G., and Bishop, G., ``An Introduction to the Kalman Filter,'' \textit{University of North Carolina Technical Report}, TR 95-041, 2006.
\bibitem{Lin2018} Lin, Y., and Saripalli, S., ``Sampling-Based Path Planning for UAVs in 3D Environments,'' \textit{IEEE Transactions on Robotics}, vol. 34, no. 3, pp. 656--670, 2018.
\bibitem{Chen2020} Chen, J., and Zhang, W., ``Spatial-Temporal Path Planning for UAV Swarms,'' \textit{Journal of Intelligent and Robotic Systems}, vol. 99, no. 2, pp. 233--245, 2020.

% Collision Avoidance

\bibitem{Khatib1986} Khatib, O., ``Real-Time Obstacle Avoidance for Manipulators and Mobile Robots,'' \textit{The International Journal of Robotics Research}, vol. 5, no. 1, pp. 90--98, 1986.
\bibitem{Zhang2021} Zhang, Z., and Zhao, S., ``UAV Collision Avoidance Using Artificial Potential Fields,'' \textit{IEEE Transactions on Control Systems Technology}, vol. 29, no. 4, pp. 1745--1756, 2021.
\bibitem{Gageik2015} Gageik, N., Benz, P., and Montenegro, S., ``Obstacle Detection and Collision Avoidance for a UAV with Complementary Low-Cost Sensors,'' \textit{IEEE Access}, vol. 3, pp. 599--609, 2015.
\bibitem{Lin2020} Lin, L., and Goodrich, M. A., ``UAV Collision Avoidance Using Reactive Strategies,'' \textit{Journal of Field Robotics}, vol. 37, no. 5, pp. 814--832, 2020.
\bibitem{Reynolds1987} Reynolds, C. W., ``Flocks, Herds, and Schools: A Distributed Behavioral Model,'' \textit{ACM SIGGRAPH Computer Graphics}, vol. 21, no. 4, pp. 25--34, 1987.
\bibitem{OlfatiSaber2006} Olfati-Saber, R., ``Flocking for Multi-Agent Dynamic Systems: Algorithms and Theory,'' \textit{IEEE Transactions on Automatic Control}, vol. 51, no. 3, pp. 401--420, 2006.
\bibitem{Ross2018} Ross, S., and Melik-Barkhudarov, N., ``Vision-Based Navigation for UAVs,'' \textit{IEEE Robotics and Automation Letters}, vol. 3, no. 4, pp. 3745--3752, 2018.
\bibitem{Wang2020} Wang, Q., and Zhang, H., ``Vision-Based Obstacle Avoidance for UAV Swarms,'' \textit{Journal of Unmanned Aerial Systems}, vol. 6, no. 2, pp. 89--97, 2020.

% Task Allocation and Coordination

\bibitem{Mnih2015} Mnih, V., et al., ``Human-Level Control through Deep Reinforcement Learning,'' \textit{Nature}, vol. 518, no. 7540, pp. 529--533, 2015.
\bibitem{Schulman2017} Schulman, J., et al., ``Proximal Policy Optimization Algorithms,'' \textit{arXiv preprint arXiv:1707.06347}, 2017.
\bibitem{Theraulaz1999} Theraulaz, G., and Bonabeau, E., ``A Brief History of Stigmergy,'' \textit{Artificial Life}, vol. 5, no. 2, pp. 97--116, 1999.
\bibitem{Beckers2000} Beckers, R., et al., ``From Local Actions to Global Tasks: Stigmergy and Collective Robotics,'' \textit{Artificial Life IV}, pp. 181--189, 2000.
\bibitem{OlfatiSaber2007} Olfati-Saber, R., Fax, J. A., and Murray, R. M., ``Consensus and Cooperation in Networked Multi-Agent Systems,'' \textit{Proceedings of the IEEE}, vol. 95, no. 1, pp. 215--233, 2007.
\bibitem{Ren2007} Ren, W., and Beard, R. W., ``Consensus Seeking in Multiagent Systems Under Dynamically Changing Interaction Topologies,'' \textit{IEEE Transactions on Automatic Control}, vol. 52, no. 5, pp. 776--781, 2007.
\bibitem{Low2019} Low, K. H., and Chen, J., ``A Virtual Navigator Model for UAV Swarm Coordination,'' \textit{IEEE International Conference on Robotics and Automation}, pp. 1234--1240, 2019.
\bibitem{Zhang2022} Zhang, X., and Liu, Y., ``Dynamic Path Planning with Virtual Navigator for UAV Swarms,'' \textit{Journal of Intelligent and Robotic Systems}, vol. 105, no. 3, pp. 45--56, 2022.

% Formation Control

\bibitem{Consolini2008} Consolini, L., et al., ``Leader-Follower Formation Control of Nonholonomic Mobile Robots with Input Constraints,'' \textit{Automatica}, vol. 44, no. 5, pp. 1343--1349, 2008.
\bibitem{Wang2019} Wang, L., and Liu, J., ``UAV Formation Control Using Leader-Follower Strategies,'' \textit{IEEE Transactions on Aerospace and Electronic Systems}, vol. 55, no. 6, pp. 3210--3221, 2019.
\bibitem{Lewis1997} Lewis, M. A., and Tan, K. H., ``High Precision Formation Control of Mobile Robots Using Virtual Structures,'' \textit{Autonomous Robots}, vol. 4, no. 4, pp. 387--403, 1997.
\bibitem{Ren2008} Ren, W., and Atkins, E., ``Distributed Multi-Vehicle Coordinated Control via Local Information Exchange,'' \textit{International Journal of Robust and Nonlinear Control}, vol. 18, no. 10, pp. 1002--1033, 2008.
\bibitem{Balch1998} Balch, T., and Arkin, R. C., ``Behavior-Based Formation Control for Multi-Robot Teams,'' \textit{IEEE Transactions on Robotics and Automation}, vol. 14, no. 6, pp. 926--939, 1998.
\bibitem{Reynolds2000} Reynolds, C. W., ``Interaction with Groups of Autonomous Characters,'' \textit{Game Developers Conference}, 2000.
\bibitem{Ren2007b} Ren, W., ``Consensus Strategies for Cooperative Control of Vehicle Formations,'' \textit{IET Control Theory and Applications}, vol. 1, no. 2, pp. 505--512, 2007.
\bibitem{Dong2016} Dong, X., et al., ``Time-Varying Formation Control for Unmanned Aerial Vehicles: Theories and Applications,'' \textit{IEEE Transactions on Control Systems Technology}, vol. 23, no. 1, pp. 340--348, 2016.
\bibitem{Zavlanos2007} Zavlanos, M. M., and Pappas, G. J., ``Distributed Formation Control with Permutation Symmetries,'' \textit{IEEE Conference on Decision and Control}, pp. 2894--2899, 2007.
\bibitem{Mesbahi2010} Mesbahi, M., and Egerstedt, M., \textit{Graph Theoretic Methods in Multiagent Networks}, Princeton University Press, 2010.

% Communication Optimization

\bibitem{Li2016} Li, Y., and Chen, H., ``Reactive-Greedy-Reactive in Unmanned Aerial Vehicle Routing,'' \textit{IEEE Communications Letters}, vol. 20, no. 8, pp. 1595--1598, 2016.
\bibitem{Zhang2018b} Zhang, J., and Zhao, T., ``RGR Protocol for Efficient Routing in FANETs,'' \textit{Ad Hoc Networks}, vol. 81, pp. 123--134, 2018.
\bibitem{Bitencourt2016} Bitencourt, J. F., et al., ``Bee-Inspired Routing Protocols for UAV Networks,'' \textit{Journal of Network and Computer Applications}, vol. 69, pp. 76--84, 2016.
\bibitem{Karaboga2012} Karaboga, D., and Ozturk, C., ``A Novel Clustering Approach: Artificial Bee Colony (ABC) Algorithm,'' \textit{Applied Soft Computing}, vol. 11, no. 1, pp. 652--657, 2012.
\bibitem{Chen2019} Chen, M., et al., ``EPOS: Energy-Efficient Planning and Optimization for UAV Swarms,'' \textit{IEEE Transactions on Mobile Computing}, vol. 18, no. 11, pp. 2598--2611, 2019.
\bibitem{Liu2021} Liu, Q., and Zhang, Y., ``Energy-Aware Task Allocation for UAV Swarms Using EPOS,'' \textit{Journal of Intelligent and Robotic Systems}, vol. 103, no. 2, pp. 34--45, 2021.
\bibitem{Zhang2020} Zhang, K., and Yang, Z., ``Graph Attention Networks for UAV Swarm Control,'' \textit{IEEE Transactions on Neural Networks and Learning Systems}, vol. 31, no. 10, pp. 4012--4023, 2020.
\bibitem{Wang2021} Wang, H., and Li, J., ``Decentralized Actor-Critic for Multi-UAV Coordination,'' \textit{IEEE International Conference on Robotics and Automation}, pp. 5678--5684, 2021.

\end{thebibliography}

\end{document}