\documentclass{article}
\usepackage{listings}
\lstset{basicstyle=\footnotesize\ttfamily}

\begin{document}

\title{Skynet/Link32: A Tactical Communication Protocol and Reference Implementation}
\author{Namdak Tonpa}
\date{June 15, 2025}
\maketitle

\begin{abstract}
Link32 is a tactical communication protocol designed for low-latency, secure,
and scalable data exchange in contested environments, supporting swarm coordination,
real-time position location information (PLI), command and control (C2), and tactical
chat over UDP-based multicast networks. We present its design and reference implementation.
\end{abstract}

\tableofcontents

\section{Introduction}
Link32 is a tactical communication protocol inspired by standards such as VMF, LINK16,
TSM, SRW, and MQTT, tailored for military applications requiring robust, low-latency,
and secure data exchange in contested environments. It facilitates swarm coordination,
real-time position location information (PLI), command and control (C2), and tactical
chat using UDP-based multicast networks. Skynet is the reference server implementation
of Link32, developed in C99 with minimal dependencies to ensure portability and
 performance on resource-constrained devices.

\subsection{Properties}
Link32 and Skynet are designed with the following properties:
\begin{itemize}
    \item \textbf{Implementation}: Written in C99 for portability and performance.
    \item \textbf{Message Size}: Minimum 32-byte header (48 bytes with AES-256-GCM tag) to optimize bandwidth.
    \item \textbf{Security}: ECDH key exchange over secp384r1, AES-256-GCM encryption for all messages.
    \item \textbf{Latency}: Microsecond-precision timing using monotonic clocks and non-blocking I/O.
    \item \textbf{Concurrency}: Lock-free atomic operations (CMPXCHG) for thread-safe queue management.
    \item \textbf{Networking}: UDP multicast with topic-based subscriptions.
    \item \textbf{Footprint}: $\sim$64KB L1 cache usage, $\sim$2000 lines of code (LOC).
    \item \textbf{Dependencies}: Single dependency on OpenSSL for cryptography.
    \item \textbf{Threat Model}: Prioritizes confidentiality and integrity, no non-repudiation.
\end{itemize}

\subsection{Principles}
Link32 adheres to the following design principles:
\begin{itemize}
    \item \textbf{Key Provisioning}: Manual public key distribution for controlled setup.
    \item \textbf{Mandatory Encryption}: All messages encrypted with AES-256-GCM.
    \item \textbf{Node Identification}: Node names hashed to 32-bit using FNV-1a.
    \item \textbf{Lock-Free Design}: Uses atomic compare-and-swap (CMPXCHG) for concurrency.
    \item \textbf{Topic Architecture}: Topics map to IP multicast groups for publish-subscribe.
    \item \textbf{Queue Management}: Global network queue with per-topic subscriber queues.
    \item \textbf{Key Storage}: Separate key stores per executable.
\end{itemize}

\section{Link32 Protocol}

\subsection{S-Message Format}
The Link32 message structure, \texttt{SkyNetMessage}, is compact for large-scale swarm communication:
\begin{lstlisting}
typedef struct {
    uint8_t version : 4;   // Protocol version (current: 1)
    uint8_t type : 4;      // Message type (0-6)
    uint8_t qos : 4;       // Quality of Service (0-3)
    uint8_t hop_count : 4; // Hop count for routing (0-15)
    uint32_t npg_id;       // Topic identifier (1-103)
    uint32_t node_id;      // Sender node ID (FNV-1a hash)
    uint32_t seq_no;       // Sequence number for deduplication
    uint8_t iv[16];        // AES-256-GCM initialization vector
    uint16_t payload_len;  // Payload length (0-32767)
    uint8_t payload[MAX_BUFFER]; // Encrypted payload + 16-byte GCM tag
} SkyNetMessage;
\end{lstlisting}
\begin{itemize}
    \item \textbf{Header Size}: 32 bytes.
    \item \textbf{Total Size}: 48 bytes minimum (32-byte header + 16-byte GCM tag).
    \item \textbf{Payload}: Up to 32720 bytes.
\end{itemize}

\subsection{Message Types}
The protocol defines seven message types, as shown in Table~\ref{tab:msgtypes}:
\begin{table}[h]
\centering
\caption{Link32 Message Types}
\begin{tabular}{cll}
\hline
\textbf{ID} & \textbf{Type} & \textbf{Description} \\
\hline
0 & Key Exchange & Exchanges ECC public keys for ECDH session setup. \\
1 & Slot Request & Requests a TDMA slot from the server. \\
2 & Chat & Sends tactical chat messages. \\
3 & Ack & Acknowledges slot assignments or control messages. \\
4 & Waypoint & Specifies navigation waypoints for C2. \\
5 & Status & Reports position, velocity, or sensor data (e.g., PLI). \\
6 & Formation & Coordinates swarm formations. \\
\hline
\end{tabular}
\end{table}

\subsection{Multicast Topics}
Link32 uses multicast topics mapped to IP multicast groups, as listed in Table~\ref{tab:topics}:
\begin{table}[h]
\centering
\caption{Multicast Topics}
\begin{tabular}{clll}
\hline
\textbf{NPG} & \textbf{Name} & \textbf{Multicast} & \textbf{Purpose} \\
\hline
1 & npg\_control & 239.255.0.1 & Network control (key exchange, slot requests). \\
6 & npg\_pli & 239.255.0.6 & Position information (status messages). \\
7 & npg\_surveillance & 239.255.0.7 & Sensor data forwarding. \\
29 & npg\_chat & 239.255.0.29 & Tactical chat and acknowledgments. \\
100 & npg\_c2 & 239.255.0.100 & Command and control (waypoints, formations). \\
101 & npg\_alerts & 239.255.0.101 & Network alerts and self-healing. \\
102 & npg\_logistics & 239.255.0.102 & Logistical coordination (status, chat). \\
103 & npg\_coord & 239.255.0.103 & Inter-agent coordination (chat, waypoints). \\
\hline
\end{tabular}
\end{table}

\subsection{Slot Management}
Link32 employs a Time Division Multiple Access (TDMA)-like slot manager to minimize collisions:
\begin{itemize}
    \item \textbf{Slot Array}: Fixed-size array (\texttt{slots[SLOT\_COUNT=256]}) in \texttt{ServerState}.
    \item \textbf{Dynamic Topics}: Each slot creates a temporary multicast group (239.255.1.<slot\_id \% 256>).
    \item \textbf{Allocation}: First-come, first-serve with no timeouts.
    \item \textbf{Timing}: Slots cycle every \texttt{TIME\_SLOT\_INTERVAL\_US=1000$\mu$s}.
\end{itemize}
Clients send \texttt{SKYNET\_MSG\_SLOT\_REQUEST} to NPG 1, and the server assigns slots via \texttt{SKYNET\_MSG\_ACK}.

\subsection{Deduplication}
A fixed-size circular buffer (\texttt{seq\_cache}) prevents message loops:
\begin{itemize}
    \item \textbf{Structure}: Stores \texttt{\{node\_id, seq\_no, timestamp\}}.
    \item \textbf{Memory}: $\sim$16KB (1024 $\times$ 16 bytes).
    \item \textbf{Complexity}: O(1) lookup using FNV-1a hashing.
    \item \textbf{Threshold}: Discards duplicates within 1 second.
\end{itemize}

\subsection{Security}
\label{subsec:security}
Security mechanisms include:
\begin{itemize}
    \item \textbf{Key Exchange}: ECDH over secp384r1 for 256-bit AES keys.
    \item \textbf{Encryption}: AES-256-GCM with 16-byte IV and 16-byte tag.
    \item \textbf{Key Storage}: Server: \texttt{\textasciitilde/.skynet/ecc/secp384r1/<node\_hash>.\{ec\_priv,ec\_pub\}}; Client: \texttt{\textasciitilde/.skynet\_client/ecc/secp384r1/<node\_hash>.\{ec\_priv,ec\_pub\}}.
    \item \textbf{Key Derivation}: HKDF-SHA256 for AES keys.
    \item \textbf{Self-Sent Handling}: Skips messages where \texttt{msg->node\_id == state->node\_id}.
\end{itemize}

\subsection{Subscriptions}
\label{subsec:subs}
Nodes subscribe to topics based on roles, as shown in Table~\ref{tab:subs}:
\begin{table}[h]
\centering
\caption{Role-Based Subscriptions}
\begin{tabular}{llp{8cm}}
\hline
\textbf{Role} & \textbf{NPGs} & \textbf{Purpose} \\
\hline
Infantry & 1, 29 & Network control and tactical chat. \\
Drone & 1, 6, 7, 100, 101 & Control, PLI, surveillance, C2, alerts. \\
Air & 1, 6, 7, 100, 101, 103 & Control, PLI, surveillance, C2, alerts, coordination. \\
Sea & 1, 7, 29, 102, 103 & Control, surveillance, chat, logistics, coordination. \\
Ground & 1, 7, 29, 102 & Control, surveillance, chat, logistics. \\
Relay & 1, 6, 101 & Control, PLI, alerts for relaying. \\
Controller & 1, 6, 100, 101 & Control, PLI, C2, alerts for command posts. \\
\hline
\end{tabular}
\end{table}

\section{Skynet Implementation}

\subsection{Dependencies}
\begin{itemize}
    \item \textbf{OpenSSL}: For ECC, ECDH, and AES-256-GCM.
    \item \textbf{C99 Compiler}: GCC or equivalent.
    \item \textbf{POSIX Environment}: For threading, epoll, and timerfd.
\end{itemize}

\subsection{Build}
To build Skynet:
\begin{lstlisting}[language=bash]
$ git clone git@github.com:BitEdits/skynet
$ cd skynet
$ gcc -o skynet_client skynet_client.c skynet_proto.c -lcrypto
$ gcc -o skynet skynet.c skynet_proto.c -pthread -lcrypto
\end{lstlisting}

\subsection{Installation}
The provisioning script \texttt{skynet.sh} generates ECC key pairs:
\begin{lstlisting}[language=bash]
# ./skynet.sh
Generated keys for node npg_control (hash: 06c5bc52) in /secp/
Generated keys for node npg_pli (hash: c9aef284) in /secp/
Generated keys for node npg_surveillance (hash: 4d128cdc) in /secp/
Generated keys for node npg_chat (hash: 9c69a767) in /secp/
Generated keys for node npg_c2 (hash: 89f28794) in cc/secp/
Generated keys for node npg_alerts (hash: 9f456bca) in /secp/
Generated keys for node npg_logistics (hash: 542105cc) in /secp/
Generated keys for node npg_coord (hash: e46c0c22) in /secp/
Generated keys for node server (hash: 40ac3dd2) in /secp/
Generated keys for node client (hash: 8f929c1e) in /client/secp/
# cp /secp/*.ec_pub /client/secp/
\end{lstlisting}

\subsection{Server Operation}
The server binds to UDP port 6566, joins multicast groups, and processes messages using a global queue. Example output:
\begin{lstlisting}
# skynet server
Node 40ac3dd2 bound to 0.0.0.0:6566.
Joined multicast group 239.255.0.1 (NPG 1: control).
Joined multicast group 239.255.0.6 (NPG 6: PLI).
Message received, from=8f929c1e, to=1, size=231.
Decryption successful, from=8f929c1e, to=1, size=215.
Saved public key for client 8f929c1e.
Assigned slot 0 to node 8f929c1e.
Message received, from=8f929c1e, to=6, size=40.
Decryption successful, from=8f929c1e, to=6, size=24.
Message sent from=8f929c1e, to=6, seq=3, multicast=239.255.1.0, latency=36643.
\end{lstlisting}

\subsection{Client Operation}
The client joins topic-specific multicast groups and sends key exchange, slot requests, and status messages. Example output:
\begin{lstlisting}
# skynet_client client
Node 8f929c1e connecting to port 6566.
Joined multicast group 239.255.0.1 (NPG 1).
Joined multicast group 239.255.0.6 (NPG 6).
Sent key exchange message to server.
Sent slot request message to server.
Received slot assignment: slot=0.
Joined slot multicast group 239.255.1.0.
Sent status message: multicast=239.255.1.0,
                     pos=[0.1, 0.1, 0.1],
                     vel=[0.0, 0.0, 0.0],
                     seq=2.
\end{lstlisting}

\subsection{Usage}
Skynet includes five utilities:

\subsubsection*{Keys Provisioning}
Generates ECC key pairs.
\begin{lstlisting}
skynet_keygen <node> [--server|--client]
\end{lstlisting}

\subsubsection*{Message Encryption}
Encrypts a test message to \texttt{<npg\_id>.sky}.
\begin{lstlisting}
skynet_encrypt <sender> <recipient> <file>
\end{lstlisting}

\subsubsection*{Message Decryption}
Decrypts \texttt{<file.sky>}.
\begin{lstlisting}
skynet_decrypt <sender> <recipient> <file.sky>
\end{lstlisting}

\subsubsection*{Skynet Server}
Runs the server with FNV-1a hashed \texttt{<node>}.
\begin{lstlisting}
skynet <node>
\end{lstlisting}

\subsubsection*{Skynet Client}
Runs the client with FNV-1a hashed \texttt{<node>}.
\begin{lstlisting}
skynet_client <node>
\end{lstlisting}

\subsection{Limitations}
\begin{itemize}
    \item \textbf{Slot Scalability}: Fixed \texttt{SLOT\_COUNT=256} limits nodes to 256.
    \item \textbf{No Retransmission}: Dropped messages are not retransmitted.
    \item \textbf{Key Management}: Manual public key copying required.
    \item \textbf{Deduplication}: \texttt{SEQ\_CACHE\_SIZE=1024} may cause collisions.
\end{itemize}

\section{Conclusion}
Link32 and Skynet provide a robust framework for tactical communication, combining low-latency,
security, and scalability. Future improvements could address slot scalability, automated key
distribution, and enhanced deduplication to support larger networks.

\begin{thebibliography}{10}
\bibitem{milstd6016} U.S. Department of Defense, ``MIL-STD-6016: Tactical Data Link (TDL) J Message Standard,'' 2008.
\bibitem{ada404334} U.S. Department of Defense, ``Link 16 Network Management and Operations,'' ADA404334, 2003.
\end{thebibliography}

\end{document}
